\documentclass{article}
\usepackage[margin=1in]{geometry}
\usepackage[utf8]{inputenc}
\usepackage[spanish, mexico]{babel}

\usepackage{setspace}
\usepackage{dirtytalk}

\title{Trabajo Bases de Datos: Evaluación de la habitabilidad de edificios afectados por sismo utilizando la teoría de conjuntos difusos y las redes neuronales artificiales.}
\author{Miguel Angel Gómez Barrera}

\begin{document}
	\maketitle
\doublespacing
Los autores M.L Carreño, A.H. Barbat y O.D. Cardona, investigadores en el area de riesgos, proponen un modelo basado en conjuntos difusos y redes neuronales artificiales para determinar si posterior a un sismo una edificación es habitable y si tiene reparación, debido a que en eventos sísmicos no es posible tener el número necesario de expertos para evaluar las condiciones de estos edificios o no es práctico utilizar un método detallado. El modelo utiliza la idea de pertenencia al conjunto (el conjunto de grado de daño), en base al juicio de una persona en el sitio y una red neuronal, se determinan los dos criterios: habitabilidad y reparabilidad, de esta manera se hace una predicción del nivel de riesgo. Los autores evalúan los resultados con el terremoto en Quindío Colombia (1999) y con fotos e informes, demuestran que el esquema propuesto puede funcionar, sin embargo advierten que en estos momentos para utilizar el sistema se requiere una mayor cantidad de información (no se indica qué tanta) para predecir con mayor certeza. Esto es importante porque en un país como el nuestro, se necesitan tomar decisiones acertadas que se traduzcan en mejores inversiones, una aproximación mas 'inteligente' a la solución en este tipo de emergencias, estos modelos ayudarían a tener una idea general y en base a ello decidir en donde enfocar los esfuerzos y el presupuesto de la nación. En mi opinión personal, este artículo es mucho más completo dado que utiliza la idea de conjunto difuso pero aquí se combina con computación para ayudar a resolver un problema complejo.

\nocite{*}
\bibliographystyle{unsrt}
\bibliography{bibliography}

\end{document}