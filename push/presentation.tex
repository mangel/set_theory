\documentclass{beamer}

\usepackage[utf8]{inputenc}
\usepackage[spanish, mexico]{babel}

\title{Ejercicio Teoría de Conjuntos II}
\author{Miguel Angel Gomez Barrera}
\institute{Fundación Universitaria Konrad Lorenz}
\date{2020}

\begin{document}
	\frame{\titlepage}
	\begin{frame}
		\frametitle{Ejercicio}
		Sean $X$ y $Y$ dos conjuntos en los cuales existe una función $f$ desde $P(X)$ a $P(Y)$, que a cada subconjunto $A$ de $X$ asigna una imagen del subconjunto $f(X)$ de $Y$. Y sea $f^{-1}$ el inverso de $f$, una función desde $P(Y)$ a $P(X)$, tal que $B$ es subconjunto de $Y$
		Demostrar:
		\begin{itemize}
			\item Si $B \subset Y$, entonces $ f(f^{-1}(B))\subset B$.
			\item Si $f$ va desde $X$ a $Y$, entonces $f(f^{-1}(B)) = B$.
			\item Si $A \subset X$, entonces $A \subset f^{-1}(f(A))$.
			\item Si $f$ es uno a uno entonces $A = f^{-1}(f(A))$.
		\end{itemize}
	\end{frame}
	\begin{frame}
		\frametitle{Demostración de la propiedad.}
		Tener presente que $f^{-1} (B) = \{ x \in X: f(x) \in B\}$
		\begin{block}{Propiedad}
			Si $B \subset Y$, entonces $ f(f^{-1}(B))\subset B$.
			\begin{proof}
				Si $y \in f(f^{-1}(B))$, entonces $y= f(x)$ para algún $x$ en $f^{-1}(B)$, lo que implica que $y = f(x)$ y $f(x) \in B$ y por ende $y \in B$.
			\end{proof}
		\end{block} 
	\end{frame}
	\begin{frame}
		\frametitle{Demostración de la propiedad.}
		Tener presente que $f^{-1} (B) = \{ x \in X: f(x) \in B\}$
		\begin{block}{Propiedad}
			Si $f$ va desde $X$ a $Y$, entonces $f(f^{-1}(B)) = B$.
			\begin{proof}
				Si $y \in B$, entonces $y = f(x)$ para algún $x$ en $X$, y por tanto para algún $x$ en $f^{-1}(B)$; ello implica que $y \in f(f^{-1}(B))$.
			\end{proof}
		\end{block} 
	\end{frame}
	\begin{frame}
		\frametitle{Demostración de la propiedad.}
		\begin{block}{Propiedad}
			Si $A \subset X$, entonces $A \subset f^{-1}(f(A))$.
			\begin{proof}
				Si $x \in A$, entonces $f(x) \in f(A)$; Lo que implica que $x \in f^{-1}(f(A))$.
			\end{proof}
		\end{block} 
	\end{frame}
	\begin{frame}
		\frametitle{Demostración de la propiedad.}
		\begin{block}{Propiedad}
			Si $f$ es uno a uno entonces $A = f^{-1}(f(A))$.
			\begin{proof}
				Si $x \in f^{-1}(f(A))$, entonces $f(x) \in f(A)$ y por ende $f(x) = f(u)$ para algún $u$ en $A$; ello implica que $x = u$ y en consecuencia que $x \in A$.
			\end{proof}
		\end{block} 
	\end{frame}
\end{document}