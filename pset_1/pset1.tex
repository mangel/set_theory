\documentclass{article}

\usepackage{amsmath}
\usepackage{amsfonts}
\usepackage{amssymb}
\usepackage[margin=0.5in]{geometry}
\usepackage[spanish, mexico]{babel}
\usepackage[utf8]{inputenc}

\title{Taller 1}
\author{Miguel A. Gomez B.}

\begin{document}
	\maketitle
	
	\paragraph{I} Para cada una de las las siguientes proposiciones:(A) simbolice cada una,(B) simbolice su negación, (C) escriba en correcto español su negación.
	\paragraph{1} Tienes clase a las 7 y a las 11.
	\paragraph{2} Si tienes clase el martes, tienes clase el jueves.
	\paragraph{3} Ramiro está inscrito en Matemáticas Básicas o en Inglés 3.
	\paragraph{4} No estás inscrito en este curso.
	\paragraph{5} Si cursaste Matemáticas Básicas y tu P.A.P.A. es superior a 3.0, puedes tomar un curso de Estadística el próximo semestre.
	\paragraph{6} Eres un estudiante de la Facultad de Medicina o de la Facultad de Odontología pero no de ambas.
	\paragraph{II} Considere las proposiciones. \textit{m}: apruebas todas las materias, \textit{p}: tienes unpromedio mayor de 3.5, \textit{s}: pasas el semestre y, \textit{b}: obtienes una beca. Simbolice las siguientes proposiciones:
	\paragraph{1} Pasas el semestre y no apruebas todas las materias.
	\paragraph{2} Tienes un promedio mayor a 3.5 o apruebas todas las materias.
	\paragraph{3} Si apruebas todas las materias, pasas el semestre.
	\paragraph{4} Pasas el semestre si tienes un promedio mayor de 3.5 o apruebas todas las materias.
	\paragraph{5} Es suficiente que tengas un promedio mayor a 3.5 para que pases el semestre.
	\paragraph{6} Es necesario que apruebes todas las materias para que pases el semestre.
	\paragraph{7} No es necesario que apruebes todas las materias para que pases el semestre.
	\paragraph{8} Para tener beca es suficiente que apruebes todas las materias y que tengas un promedio mayor de 3.5.
	\paragraph{9} Si no apruebas todas las materias es necesario que tengas un promedio mayor a 3.5 para que pases el semestre.
	\paragraph{III} Si la proposición $p \land q \rightarrow r$ es falsa, determine (si es posible) el valor de verdad de lassiguientes proposiciones. Si no es posible, explique por qué:
	\paragraph{a} $(q \lor r) \land p$
	\paragraph{b} $q \land s \rightarrow p$
	\paragraph{c} $q \lor s \rightarrow p$
	\paragraph{d} $\neg p \land q \rightarrow r \land s$
	\paragraph{IV} Determine si las siguientes parejas de proposiciones son equivalentes:
	\paragraph{1}$p \rightarrow q$ y $\neg p \land q$
	\paragraph{2}$p \rightarrow q$ y $\neg p \lor q$
	\paragraph{3}$p \rightarrow q$ y $\neg q \rightarrow \neg p$
	\paragraph{4}$p \rightarrow q$ y $\neg p \rightarrow q$
\end{document}