\documentclass{beamer}

\usepackage[utf8]{inputenc}
\usepackage[spanish, mexico]{babel}

\title{Ejercicio Teoría de Conjuntos}
\author{Miguel Angel Gomez Barrera}
\institute{Fundación Universitaria Konrad Lorenz}
\date{2020}

\begin{document}
	\frame{\titlepage}
	\begin{frame}
		\frametitle{Ejercicio}
		Muestre que para cualquier conjunto $X$, $\bigcap \mathcal{P}(X) = \emptyset$.	
	\end{frame}
	\begin{frame}
		\frametitle{Prueba.}
		\begin{block}{}
			La demostración es de manera directa, primero demostraremos que
			\begin{block}{Propiedad}
				Para todo conjunto $A$, $A \cap \emptyset = \emptyset.$
			\end{block} 
		\end{block}
	\end{frame}
	\begin{frame}
		\frametitle{Demostración de la propiedad.}
		\begin{block}{Propiedad}
			Para todo conjunto $A$, $A \cap \emptyset = \emptyset.$
			\begin{proof}
				Consideremos $x \in A \cap \emptyset$, por definición no pertenece al vacío y en general para todo $x \in A$ esto se cumple, luego $A \cap \emptyset = \emptyset$, aún si consideramos que $A = \emptyset$ por definición del conjunto vacío la intersección tampoco tendrá elementos.
			\end{proof}
		\end{block} 
	\end{frame}
	\begin{frame}
		\frametitle{Un resultado adicional}
		\begin{block}{Ejercicio}
			$\{\emptyset\} \subseteq A$.
			\begin{proof}
				$\{\emptyset\} \subseteq A$. Suponga que no es así, que en $\emptyset$ hay algún $x \notin A$, ello es un contradicción dado que el vacío no tiene elementos, por lo tanto la proposición es verdadera.
			\end{proof}
		\end{block}
	\end{frame}
	\begin{frame}
		\frametitle{Prueba cont.}
		\begin{block}{}
			Muestre que para cualquier conjunto $X$, $\bigcap \mathcal{P}(X) = \emptyset$.	
			\begin{proof}
				Para todo conjunto $X$, $\emptyset \in \mathcal{P}(X)$, por las demostraciones anteriores vemos que $\bigcap \mathcal{P}(X) = \emptyset.$
			\end{proof}
		\end{block}
	\end{frame}
\end{document}